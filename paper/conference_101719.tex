\documentclass[conference]{IEEEtran}
\IEEEoverridecommandlockouts
% The preceding line is only needed to identify funding in the first footnote. If that is unneeded, please comment it out.
\usepackage{cite}
\usepackage{amsmath,amssymb,amsfonts}
\usepackage{algorithmic}
\usepackage{graphicx}
\usepackage{textcomp}
\usepackage{xcolor}
\def\BibTeX{{\rm B\kern-.05em{\sc i\kern-.025em b}\kern-.08em
    T\kern-.1667em\lower.7ex\hbox{E}\kern-.125emX}}
\begin{document}

\title{Impact of low-level image attributes on face detector accuracy}
% https://github.com/cvaugh/face-detection-project

\maketitle

\begin{abstract}
This document is a model and instructions for \LaTeX.
This and the IEEEtran.cls file define the components of your paper [title, text, heads, etc.]. *CRITICAL:
Do Not Use Symbols, Special Characters, Footnotes, or Math in Paper Title or Abstract.
\end{abstract}

\begin{IEEEkeywords}
%component, formatting, style, styling, insert
\end{IEEEkeywords}

\section{Introduction}

% What is the purpose of this work? 

% Add context to the story 
% - why do we care
% - why this is important 
% - summary of what we're about to tell you


\section{Related work} % lit review - roughly a page or so 

% start by reading a survey of face detectors 
% give the author a taxonomy of face detection models/methods
% cite papers that address detector weasknesses 

\section{Dataset} % GeoFaces

This dataset~\cite{geofacial}.

% talk about any particular things that images went through 
% from collection to use

\section{Experimental setup}

% this is what we we're interested in 
% here's how we tested our hypotheses

\section{Results}
% numbers and graphs

\section{Discussion}

% weaknesses 

\section{Conclusion}
% what we did
% what needs to be done (future work)

% max 8 pages
\bibliographystyle{ieeetr}
\bibliography{bibliography}

\end{document}
